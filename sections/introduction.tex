\documentclass[../main.tex]{subfiles}
\begin{document}
\newpage
\section*{Введение}
\addcontentsline{toc}{section}{Введение}
\textbf{Актуальность темы.} В теории управления хорошо развиты методы, разработанные для линейных систем. В ряде задач оптимального управления для линейных систем удается получать аналитические решения. В случае же нелинейных систем, аналитическое решение является скорее исключением, чем правилом. Поэтому в задачах управления нелинейными системами часто применяются решения, найденные в линеаризованной постановке. Иногда такой подход может быть строго обоснован. Например, согласно теореме об устойчивости по первому приближению(см., например\footnote{Барбашин Е.А. Функции Ляпунова. М.:Наука,1970, 240 c.}), из устойчивости линеаризации в окрестности положения равнвесия следует локальная устойчивость исходной нелинейной системы. При решении задачи стабилизации это позволяет приближенно заменять нелинейную систему ее линеаризацией в окрестности положения равновесия. И если линеаризованная система окажется вполне управляемой (стабилизируемой), то линейная обратная связь, стабилизирующая эту систему, будет локально (в некоторой окрестности положения равновесия) стабилизировать и нелинейную систему\footnote{Красовский Н.Н. Проблемы стабилизации управляемых движений. Дополнение редактора к книге И.Г.Малкина <<Теория устойчивости  движения>>. М.: Наука, 1966, с. 475-514.}\footnote{Альбрехт Э.Г., Шелементьев Г.С. Лекции по теории стабилизации, Уральский государственный университет им. А.М. Горького, 1972, 273 с.}\footnote{Халил Х.К. Нелинейные системы. 3 изд. М.-Ижевск. НИЦ <<Регулярная и хаотическая динамика>>, Институт компьютерных исследований, 2009, 832 стр.}\footnote{Поляк Б.Т., Хлебников М.В., Рапопорт Л.Б., Математическая теория автоматического управления: учебное пособие. М. Ленанд, 2019, 500 с.}.  Однако зачастую, метод линеаризации применяется без должного обоснования, так как соответсвующие условия либо сложны для проверки, либо вообще отсутствуют.

Данная работа посвящена изучению свойств множеств достижимости для аффинно-управляемых систем, описываемых обыкновенными дифференциальными уравнениями, с интегральными ограничениями на управление. В первом разделе показывается, как на основе условия выпуклости нелинейного шара в гильбертовом пространстве получить условия выпуклости множеств достижимости рассматриваемых нелинейных систем на малых интервалах времени. Проверка этих условий требует изучения асимптотики наименьшего собственного числа грамиана управляемости линеаризованной системы. Один из возможных способов проверки асимптотики собственных чисел грамиана предлагается во втором разделе. Там же приводится доказательство выпуклости множеств достижимости на малых интервалах времени для некоторых классов нелинейных систем второго порядка. В третьем разделе исследуются множества достижимости нелинейных систем по выходу на малых интервалах времени. Используя понятие асимптотической эквивалентности\footnote{Goncharova E., Ovseevich A. Small-time reachable sets of linear systems with integral control constraints: birth of the shape of a reachable set, Journal of Optimization Theory and Applications,
2016, vol. 168, no. 2, pp. 615–624. https://doi.org/10.1007/s10957-015-0754-4}, основанное на расстоянии Банаха-Мазура, доказывается близость множеств достижимости нелинейных систем по выходу соответсвующим множествам достижимости линеаризованных систем на малых интервалах времени. В четвертом разделе выводятся условия применимости метода линеаризации в задаче локального синтеза. Эти условия совпадают с условием асимптотической эквивиалентности множеств достижимости нелинейной и линеаризованной систем.

\textbf{Цель работы.} Изучение поведения множеств достижимости нелинейных систем с интегральными ограничениями на малых интервалах времени исследование их взаимосвязи с множествами достижимости линеаризованных систем.
	
\textbf{Методы исследования.} Предлагаемые исследования основаны на результатах теории дифференциальных уравнений и математической теории управления, нелинейном и выпуклом анализах. В работе исследуется условие выпуклости нелинейного отображения малого шара в гильбертовом пространстве. Исследования на эту тему были инициированы Б.Т. Поляком\footnote{Polyak, B.T. Convexity of Nonlinear Image of a Small Ball with Applications to Optimization. Set-Valued Analysis 9, 159–168 (2001). https://doi.org/10.1023/A:1011287523150} и продолжены другими авторами, например, Ю.С. Ледяевым\footnote{Ю. С. Ледяев, “Критерии выпуклости замкнутых множеств в банаховых пространствах”, Оптимальное управление и дифференциальные уравнения, Сборник статей. К 110-летию со дня рождения академика Льва Семеновича Понтрягина, Труды МИАН, 304, МИАН, М., 2019, 205–220; Proc. Steklov Inst. Math., 304 (2019), 190–204}. Полученные результаты иллюстрируются численными примерами.
	
\textbf{Научная новизна.} Научная новизна заключается в исследовании свойств нелинейных систем и их множеств достижимости на малых интервалах времени. Проведено исследование выпуклости и асимптотики множеств достижимости по выходу аффинных по управлению систем с интегральными ограничениями. Доказано, что при выполнении определенных условий, накладываемых на собственные числа соответствующего грамиана управляемости линеаризованной системы множества достижимости нелинейной системы будут в определенном смысле близки к множествам достижимости линеаризованной системы. При исследовании задачи локального синтеза выяснено, что это условие совпадает с достаточным условием применимости метода линеаризации для задачи локального синтеза с интегральными ограничениями. То есть, при выполнении этих условий, линейная обратная связь, решающая задачу приведения линеаризованной системы в начало координат за заданное время, будет приводить в начало координат и нелинейную систему.

\textbf{Теоретическая и практическая ценность работы.} Работа носит в основном теоретический характер. Доказаны достаточные условия выпуклости множеств достижимости нелинейных систем и их асимптотической эквивалентности соответствующим множествам достижимости линеаризованных систем на малых интервалах времени. Описаны границы применения метода линеаризации в описании множеств достижимости нелинейных систем с интегральными ограничениями и в задаче локального синтеза для таких систем. Полученные результаты могут использоваться на практике для оценки применимости метода линеаризации в конкретных задачах управления.
	
\textbf{Апробация работы.}  Основные результаты, полученные в процессе исследования, докладывались и обсуждались на следующих конференциях:
\begin{enumerate}
	\item Воронежская весеняя мат. шк. «Современные методы теории краевых задач. Понтрягинские чтения–XXX», Воронеж, 3–9 мая 2019;
	\item Application of Mathematics in Technical and Natural Sciences (AMiTaNS’11): 11th Intern. Conf., June 20-25, 2019, Albena, Bulgaria;
	\item 51-я, 52-я и 53-я Международные молодежные школы-конференции "Современные проблемы математики и ее приложений"(2020, 2021, 2022 гг., Екатеринбург;
	\item III международный семинар, посвященный 75-летию академика А.И.Субботина, Екатеринбург, 26–30 октября 2020.
	\item Динамические системы: устойчивость, управление, оптимизация: материалы Междунар. науч. конф. памяти Р.Ф. Габасова, Минск, 5-10 окт. 2021;
	\item XVI Международная конференция «Устойчивость и колебания нелинейных систем управления» (конференция Пятницкого) 
	\item Теория оптимального управления и приложения (OCTA 2022),
Екатеринбург, 27 июня – 1 июля 2022 г.
	\item 7-я Международная школа-семинар  «Нелинейный анализ и экстремальные задачи» (NLA-2022), 15-22 июля, Иркутск.
\end{enumerate}
	
\textbf{Публикации.} Основные результаты исследования опубликованы в работах [1]--[7].
\end{document}