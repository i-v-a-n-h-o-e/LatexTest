\documentclass[../main.tex]{subfiles}
\begin{document}
\newpage
\addcontentsline{toc}{section}{Литература}
\section*{Литература}
\begin{enumerate}
	\item Гусев, М.И. Асимптотическое поведение множеств достижимости на малых временных промежутках / М.И.Гусев, И.О.Осипов // Труды Ин-та математики и механики. 2019. Т. 25, № 3. С. 86-99. Gusev, M.I. Asymptotic Behavior of Reachable Sets on Small Time Intervals / M.I.Gusev, I.O.Osipov // Proceedings of the Steklov Institute of Mathematics. 2020. Vol. 309, suppl.1. P. S52-S64
	\item Gusev, M.I. On Convexity of Small-time Reachable Sets of Nonlinear Control Systems / M.I. Gusev, I.O. Osipov // AIP Conference Proceedings. 2019. Vol.2164 : Application of Mathematics in Technical and Natural Sciences (AMiTaNS’11): 11th Intern. Conf., June 20-25, 2019, Albena, Bulgaria. Art. no.060007. 9 p.
	\item Гусев, М.И. Об асимптотике множеств достижимости на малых временных промежутках / М.И.Гусев, И.О.Осипов // Теория управления и теория обобщенных решений уравнений Гамильтона-Якоби: материалы III международного семинара, посвященного 75-летию академика А.И.Субботина Екатеринбург, 26–30 октября 2020. Екатеринбург: ИММ УрО РАН, 2020. С.142-145.
	\item Осипов, И. О. Об асимптотике собственных чисел грамиана управляемости линейной системы с малым параметром / И. О. Осипов // Итоги науки и техники. Сер. Соврем. математика и ее прил. Тем. обзоры. 2021. Т.191: Материалы Воронежской весенней мат. шк. «Современные методы теории краевых задач. Понтрягинские чтения–XXX», Воронеж, 3–9 мая 2019. Ч. 2. С.115–122.
	\item Осипов И.О. О выпуклости множеств достижимости по части координат нелинейных управляемых систем на малых промежутках времени / И.О. Осипов // Вестник Удмуртского университета. Серия Математика. Механика. Компьютерные науки. 2021. Т. 31. Вып. 2. C.210-225.
	\item Гусев М.И. Асимптотика множеств достижимости нелинейных управляемых систем на малых промежутках времени / М.И.Гусев, И.О.Осипов // Динамические системы: устойчивость, управление, оптимизация: материалы Междунар. науч. конф. памяти Р.Ф. Габасова, Минск, 5-10 окт. 2021. Минск: Изд. центр БГУ, 2021. С.85-87. 
	\item М. И. Гусев, И. О. Осипов. О задаче локального синтеза для нелинейных систем с интегральными ограничениями // Вестник Удмуртского университета. Математика. Механика. Компьютерные науки. 2022 (в печати)
\end{enumerate}
\begin{thebibliography}{99}
\bibitem{Albrecht1}
Albrecht~E.\,G. On the optimal motion control of quasilinear systems, \emph{Differential Equations}, 1969, Vol.5, No. 3 , P. 430-442 (in Russian)

\bibitem{Albrecht2}
Albrecht~E.\,G. The coming together of quasilinear objects in the regular case, \emph{Differential Equations}, 1971, vol. 7, no. 7, 1171-1178 (in Russian)

\bibitem{Albrecht3}
Albrecht~E.\,G. \emph{Metod Lyapunova-Puankare v zadachah optimalnogo upravleniya}. Diss.
dokt [Lyapunov-Poincare method in optimal control problems. Dr. phys. and math. sci. diss.].
Sverdlovsk, 1986. 280~p. (in Russian)

\bibitem{Calvet}
Calvet~J.-P., Arkun~Y. Design of P and PI stabilizing controllers for quasi-linear systems. \emph{Computers \& Chemical Engineering}, 1990, Vol. 14(4-5), P. 415–426. \hrefdoi{doi:10.1016/0098-1354(90)87017-j}

\bibitem{Ching}
Ching~Sh., Eun~Yo., Gokcek~C., Kabamba~P., Meerkov~S. \emph{Quasilinear Control: Performance Analysis and Design of Feedback Systems with Nonlinear Sensors and Actuators}, 2010 \hrefdoi{10.1017/CBO9780511976476}. 

\bibitem{Dauer}
Dauer~J.\,P. Nonlinear perturbations of quasi-linear control systems,
\emph{Journal of Mathematical Analysis and Applications},
Volume 54, Issue 3,
1976,
Pages 717-725,
\hrefdoi{10.1016/0022-247X(76)90191-8}.

\bibitem{Fillipov}
Filippov~A.\,F.: \emph{Differential Equations with Discontinuous Righthand Sides}. Kluwer Academic Press, Boston, 1988

\bibitem{Fillipov2}
Filippov~A.\,F.: \emph{Vvedenie v teoriju differencial'nyh uravnenij} [Introduction to the theory of differential equations]. Moscow: Comkniga, 2007. 240~p. (in~Russian)

\bibitem{Gabasov}
Gabasov~R.\,F., Kalinin~A.\,I.,  Kirillova~F.\,M., Lavrinovich~L.\,I. On asymptotic optimization methods for quasilinear control systems., \emph{Trudy Instituta Matematiki i Mekhaniki URO
	RAN}, 2019, vol. 25, no. 3, pp. 62–72.

\bibitem{Gui}
Guo~Y., Kabamba~P.\,T., Meerkov~S.\,M.,  Ossareh~H.\,R., Tang~C.\,Y. Quasilinear Control of Wind Farm Power Output, \emph{IEEE Transactions on Control Systems Technology}, 2015 vol. 23, no. 4, P. 1555-1562,  \hrefdoi{10.1109/TCST.2014.2363431}.


\bibitem{GusZyk}
Gusev~M.\,I., Zykov~I.\,V. On Extremal Properties of the Boundary Points of Reachable Sets for Control Systems with Integral Constraints, \emph{Proc. Steklov Inst. Math}, 2018, Vol. 300, P. 114--125. \hrefdoi{10.1134/S0081543818020116}

\bibitem{GusevUMJ}
Gusev~M.\,I., The limits of applicability of the linearization method in calculating small-time reachable sets // \emph{Ural Mathematical Journal}. 2020. Vol. 6, No. 1. P. 71-83

\bibitem{GusOsSteklov}
Gusev~M.\,I., Osipov I.\,O.: Asymptotic behavior of reachable sets on small time intervals. \emph{Proc. Steklov Inst. Math}, 2020, Vol. 309, Suppl. 1, P. 52--64.  \hrefdoi{10.1134/S0081543820040070}

\bibitem{GusOsUdmurt} 
Gusev~M.\,I., Osipov~I.\,O.: On a local synthesis problem for nonlinear systems with integral constraints. \emph{Vestnik Udmurtskogo Universiteta. Matematika. Mekhanika. Komp’yuternye Nauki}, 2022, Vol. 32, No. 2 , P. 171–186.
\hrefdoi{10.35634/vm220202}

\bibitem{KalininLavrinovich2018}
Kalinin~A.\,I., Lavrinovich L.I. Asymptotic minimization method of the integral quadratic functional on
the trajectories of a quasilinear dynamical system. \emph{Dokl. NAN Belarusi}, 2018, vol. 62, no. 5, pp. 519–524.
\hrefdoi{10.29235/1561-8323-2018-62-5-519-524} 

\bibitem{Kiselev}
Kiselev~Yu.\,N. An asymptotic solution of the problem of time-optimal control systmes which are close to
linear ones. \emph{Soviet Math. Dokl.}, 1968, vol. 9, no. 5, pp. 1093–1097.

\bibitem{Kras}
{Krasovskii~N.\,N.} \emph{Teoriya upravleniya dvizheniem} [Theory of Control of Motion]. Moscow: Nauka, 1968. 476~p. (in~Russian)

\bibitem{Kremlev}
Kremlev~A.\,G. On the control of quasilinear system under uncertain initial conditions, \emph{Differential Equations}, 1980, Vol. 16, No. 11, P. 1967-1979. (in Russian)

\bibitem{Osipov}
Osipov~I.\,O. On the convexity of the reachable set with respect to a part of coordinates at small time intervals. \emph{Vestn. Udmurtsk. Univ. Mat. Mekh. Komp. Nauki}, 2021,  Vol. 31, No. 2, P. 210--225.
\hrefdoi{10.35634/vm210204}

\bibitem{Polyak1964}
Polyak~B.\,T. Gradient methods for solving equations and inequalities, \emph{USSR Comput. Math. Math.
	Phys.}, 1964, Vol. 4, No. 6, P. 17–32.

\bibitem{Polyak2001}
Polyak~B.\,T. Convexity of Nonlinear Image of a Small Ball with Applications to Optimization. \emph{Set-Valued Analysis}, 2001, Vol. 9, P. 159–168.\hrefdoi{10.1023/A:1011287523150}

\bibitem{Polyak2004}
Polyak~B.\,T. Convexity of the reachable set of nonlinear systems under L2 bounded controls. \emph{Dynam. Contin. Discrete Impuls. Systems Ser. A Math. Anal.}, 2004, Vol. 11, Suppl. 2-3,  P. 255–267.

\bibitem{Subbotin}
Subbotin~A.\,I. Control of motion of a quasilinear system. \emph{Differ. Uravn.,} 1967, vol. 3, no. 7, pp. 1113–1118
(in Russian).

\bibitem{Patent} 
Zykov~I.\,V., Osipov I.O. A program for constructing the reachable sets of nonlinear systems with integral control constraints by the Monte Carlo method, certificate of state registration of a computer program No. 2020661557, 2020.

\bibitem{Zykov}
Zykov~I.\,V. An Algorithm for Constructing Reachable Sets for Systems with Multiple Integral Constraints  // {\textit Mathematical Analysis With Applications : Intern. Conf. CONCORD-90}, Ekaterinburg, July 2018. Cham : Springer, 2020. P. 51-60. (Springer Proceedings in Mathematics \& Statistics; vol. 318. \hrefdoi{10.1007/978-3-030-42176-2\_6}

\bibitem{Zykov2019}
Zykov~I.\,V., External estimates of reachable sets for control systems with integral constraints, Proceedings of the Voronezh spring mathematical school “Modern methods of the theory of boundary-value problems. Pontryagin readings – XXX”. Voronezh, May 3-9, 2019. Part 1, Itogi Nauki i Tekhniki. Ser. Sovrem. Mat. Pril. Temat. Obz., 190, VINITI, Moscow, 2021, 107–114 \hrefdoi{10.36535/0233-6723-2021-190-107-114}
\end{thebibliography}
\end{document}