\documentclass[../main.tex]{subfiles}
\begin{document}
\subsection{Выпуклость множеств достижимости нелинейных систем при малых управлениях или на малых интервалах времени} 
    В этом разделе будут описаны результаты, на которые опираются дальнейшие исследования. В первом параграфе описывается условие выпуклости нелинейного отображения малого шара в гильбертовом пространстве, полученное  Б.Т. Поляком\footnote{Polyak, B.T. Convexity of Nonlinear Image of a Small Ball with Applications to Optimization. Set-Valued Analysis 9, 159–168 (2001). https://doi.org/10.1023/A:1011287523150}. Во втором параграфе описывается применение этого условия для обоснования выпуклости множеств достижимости нелинейных систем при ограничении на управление, заданном шаром достаточно малого радиуса. В третьем параграфе исследуются условия, при которых множества достижимости нелинейных систем с интегральными ограничениями оказываются выпуклыми на малых интервалах времени. 
    
    \subsubsection{Выпуклость отображения малого шара.}
    
    Пусть $X$, $Y$ --- гильбертовы пространства, а $f: X \rightarrow Y$ --- нелинейное отображение с липшицевой производной, определенное  на шаре $B_X(a,r) = \{x \in X: \| x - a\| \leqslant r\}$, тогда 
    \begin{gather}\label{lip_cond}
        \| f'(x) - f'(z) \| \leqslant L \| x - z \|, \quad \forall x,z \in B_X(a,r).
    \end{gather}
    Предположим, что $a$ --- регулярная точка отображения $f$, то есть линейный оператор $f'(a)$ отображает $X$ на $Y$ и тогда найдется $\nu > 0$, такое что
    \begin{gather}\label{regularity_cond}
        \| f'(a)^*y\| \geqslant \nu \| y\|, \quad \forall y \in Y.
    \end{gather}
    Например, если множества $X,Y$ --- конечномерные, $X\in\mathbb{R}^n$, $Y\in\mathbb{R}^m$, то условие \eqref{regularity_cond} выполняется, если $ \operatorname{rank} f'(a) = m$. В этом случае $\nu$ --- наименьшее сингулярное число $f'(a)$. 
    \begin{theorem}\label{PolyakTh}
        Если выполняются \eqref{lip_cond}, \eqref{regularity_cond}  и 
        \begin{gather}
            \varepsilon < \min\left\{r,\frac{\nu}{2L}\right\},
        \end{gather}
        тогда образ шара $B_X(a,\varepsilon) = \{x \in X: \| x - a\| \leqslant \varepsilon\}$ при отображении $f$ выпуклый, то есть $F = 
        \{f(x): x \in B_X(a,\varepsilon)\}$ --- выпуклое множество в $Y$. Более того, это множество сильно выпуклое и его граница состоит из образов граничных точек шара: $\partial F \subset f(\partial B_X(a,\varepsilon))$
    \end{theorem}
    
    \subsubsection{Выпуклость множеств достижимости}
    На интервале времени $ t_0 \leqslant t \leqslant \overline{T} $ рассмотрим нелинейную систему, аффинную по управлению
	\begin{gather}\label{nonlinear}
		\begin{gathered}
			\dot{x}(t)=f_1(t,x(t))+f_2(t,x(t))u(t), \qquad x(t_0) = x_0.
		\end{gathered}
	\end{gather}
	Здесь $ x \in \mathbb{R}^n $ -- вектор состояния, $ u \in \mathbb{R}^r $ -- управление,  $ \overline{T} $ --- некоторое фиксированное положительное число.
	
	
	Функции $ f_1: \mathbb{R}^{n+1} \rightarrow \mathbb{R}^{n} $, $ f_2: \mathbb{R}^{n+1} \rightarrow \mathbb{R}^{n \times r} $ предполагаются непрерывными и непрерывно-дифференциру-\\емыми по $ x $.
	Также предполагается, что функции $ f_1 $, $ f_2 $ удовлетворяют условиям
	\begin{gather*}
		\begin{gathered}
			\left\| f_1(t,x) \right\| \leqslant	l_1(t)(1 + \left\| x \right\| ), \\
			\left\| f_2(t,x) \right\| \leqslant	l_2(t),  \qquad t_0 \leqslant t \leqslant \overline{T}, \qquad   x \in \mathbb{R}^n,
		\end{gathered}
	\end{gather*}
	где $ l_1(\cdot) \in \mathbb{L}_1[t_0,\overline{T}] $, $ l_2(\cdot) \in \mathbb{L}_2[t_0,\overline{T}] $.
	
	Под $ \mathbb{L}_1[t_0,\overline{T}],  \mathbb{L}_2[t_0,\overline{T}]  $ будем понимать,  соответственно, пространства
	интегрируемых и интегрируемых с квадратом скалярных или вектор-функций  на $ [t_0,\overline{T}] $. Управление $u(t)$ будем выбирать из
	пространства $\mathbb{L}_2[t_0,\overline{T}]$ вектор-функций,  скалярное произведение в котором определено равенством
	\begin{gather*}
		\left(u(\cdot),\upsilon(\cdot) \right) = \int_{t_0}^{\overline{T}}u^{\top}(t)\upsilon(t) \, dt.
	\end{gather*}
	Управление $ u(\cdot) $ ограничим шаром радиуса $ \mu, \mu > 0 $
	\begin{gather}\label{constr}
		\lVert u(\cdot)\rVert^2_{\mathbb{L}_2} = \left(u(\cdot),u(\cdot) \right) \leqslant \mu^2.
	\end{gather}
	
	В условиях описанных предположений, каждому $ u(\cdot) \in \mathbb{L}_2 $ соответствует единственное абсолютно непрерывное решение $ x(t)=x(t,u(\cdot)) $ системы \eqref{nonlinear}, определённое на интервале $ [t_0,\overline{T}] $.
	
	Все траектории $ x(t) $ системы \eqref{nonlinear}, отвечающие удовлетворяющим \eqref{constr} управлениям,  лежат внутри некоторого компактного множества $ D \subset \mathbb{R}^n $.
	
	Дадим здесь несколько определений, которыми будем пользоваться в дальнейшем.
	
	Пусть $ 0 <  T \leqslant \overline{T} $.
	\begin{definition}
		{\it Множеством достижимости} $ G(T,\mu) $ системы \eqref{nonlinear} в пространстве состояний в момент времени $ T $ назовем множество всех концов траекторий $ x(T) \in \mathbb{R}^n $,  которые могут быть порождены управлениями	$ u(t) \in B_{\mathbb{L}_2}(0,\mu) =\left\lbrace u:\lVert u(\cdot)\rVert^2_{\mathbb{L}_2} \leqslant \mu^2\right\rbrace  $,
		\begin{gather*}
			G(T,\mu)=\{x\in \mathbb{R}^n:\exists u(\cdot)\in B_{\mathbb{L}_2}(0,\mu),\; x=x(T,x^0,u(\cdot))\}.
		\end{gather*}
	\end{definition}
	\begin{definition}
		Пусть $ x(t,x_0,u(\cdot)) $ --- движение, отвечающее управлению $ u(\cdot)$. Тогда назовем систему
		\begin{gather}\label{linear}
			\delta \dot{x} =  A(t) \delta x + B(t) \delta u, \qquad t_0 \leqslant t \leqslant T, \qquad \delta x(t_0) = 0,
		\end{gather}
		{\it линеаризацией} системы \eqref{nonlinear} вдоль траектории $ x(t,x_0,u(\cdot)) $, если 
		\begin{gather*}
		    A(t) = \dfrac{\partial f_1}{\partial x} (t,x(t,x_0,u(\cdot))) + \dfrac{\partial f_2}{\partial x}(t,x(t,x_0,u(\cdot))) u(\cdot), \  B(t) = f_2 (t,x(t,0)).
		\end{gather*} Здесь $ A(t) $ представляет собой матрицу Якоби функции $ f_1 + f_2 u(\cdot) $, вычисленную вдоль траектории $ x(t,u(\cdot)) $.
	\end{definition}
	\begin{definition}
	Симметричная матрица, определённая равенством
		\begin{gather*}
			W(T) = \int_{t_0}^{T}X(T,t)B(t)B^{\top}(t)X^{\top}(T,t) \, dt,
		\end{gather*}
	называется грамианом управляемости системы \eqref{linear} на интервале времени $  t_0 \leqslant t \leqslant T $.
	\end{definition}
	Здесь $ X(\tau_1,\tau_0)= \Phi(\tau_1) \Phi^{-1}(\tau_0) $, где $\Phi(t) $ --- фундаментальная матрица решений однородной системы, удовлетворяющая уравнению 
	\begin{gather*}
		\dot{\Phi}(t) = A(t) \Phi(t), \qquad \Phi(t_0) = I.
	\end{gather*}

	Система \eqref{linear} вполне управляема на  $ [t_0, T] $ тогда и только тогда, когда ее грамиан управляемости $W(T)$ --- положительно определенная матрица.
	
	При фиксированном $x_0$ введем отображение $F: \mathbb{L}_2[0,T] \rightarrow \mathbb{R}^n $ равенством $Fu(\cdot) = x(T,u(\cdot))$: здесь $ x(T,u(\cdot))$ --- решение системы \eqref{nonlinear}, порожденное управлением $u(\cdot)$. 
	
	Это отображение имеет непрерывную производную Фреше, определяемую равенством $ F'(u(\cdot))\delta u(\cdot) = \\=\delta x(T)$, где $\delta x(T)$ --- решение линеаризованной вдоль пары $\left( x(t,u(\cdot)),u(\cdot)\right)  $ системы \eqref{linear}, порожденное управлением $\delta u(\cdot)$ при нулевых начальных условиях.
	
	 Тогда, управляемость системы \eqref{linear} означает регулярность отображения $F$ при $u = 0$, а множество достижимости $G(T,\mu)$ есть образ гильбертова шара $B_{\mathbb{L}_2}(0,\mu)$ при его отображении $F$.
	
	Применяя теорему \ref{PolyakTh} к отображению $F$, получим, что в условиях описанных преположений относительно функций $f_1$ и $f_2$, множество достижимости $G$ выпукло при достаточно малых $\mu > 0$.
	
    \subsubsection{Выпуклость множеств достижимости на малых интервалах времени}
    Рассмотрим здесь нелинейную систему \eqref{nonlinear} на малом интервале времени $\ t_0 \leqslant t \leqslant t_0 + \overline{\varepsilon} $.
    \begin{gather}\label{nonlinearT}
			\dot{x}(t)=f_1(t,x(t))+f_2(t,x(t))u(t), \qquad x(t_0) = x_0.
	\end{gather}
    Здесь $ \overline{\varepsilon} $ --- некоторое фиксированное положительное число, а кроме принятых в предыдущем разделе допущений относительно функций $f_1$ и $f_2$, которые предполагаются справедливыми на интервале $t_0 \leqslant t \leqslant t_0 + \overline{\varepsilon} $ введем следующее.
    \begin{assumption}\label{Pred}
    	Функции $f_1(t,x)$ и $f_2(t,x)$ имеют непрерывные производные по $x$, которые удовлетворяют условию Липшица при всех $t \in [t_0;t_0+\varepsilon]$,$x_1, x_2 \in D$.
    	\begin{gather*}
    			\left\| \frac{\partial f_1}{\partial x}(t,x_1) - \frac{\partial f_1}{\partial x}(t,x_2) \right\| \leqslant L_3 \| x_1 - x_2\|, \quad 	\left\| \frac{\partial f_2}{\partial x}(t,x_1) - \frac{\partial f_2}{\partial x}(t,x_2) \right\| \leqslant L_4 \| x_1 - x_2\|,
    	\end{gather*}
    	где $L_i \geqslant 0$ для $i = 3,4$.
    \end{assumption}

     Управление $u(\cdot)$ будем выбирать из
    пространства интегрируемых с квадратом функций $\mathbb{L}_2[t_0,t_0+\bar{\varepsilon}]$ и ограничим шаром радиуса $ \mu > 0 $ в этом пространстве
    \begin{gather*}
    	\lVert u(\cdot)\rVert^2_{\mathbb{L}_2} = \left(u(\cdot),u(\cdot) \right) \leqslant \mu^2.
    \end{gather*}
	Пусть $ 0 <  \varepsilon \leqslant \bar{\varepsilon} $.  
	
	Далее, используя замену времени, мы сведем задачу описания множества достижимости на малом интервале к аналогичной задаче на фиксированном интервале для системы, уравнения которой и интегральные ограничения на управления зависят от малого параметра.
    Произведя замену времени
    $ t = \varepsilon \tau + t_0 $ и приняв обозначения $ z(\tau) = x(\varepsilon \tau + t_0) $ и $ \upsilon(\tau) = \varepsilon u(\varepsilon \tau + t_0) $  получим из \eqref{nonlinearT}
        \begin{equation}\label{epsnonlinear}
    	\dot{z}(\tau)=\widetilde{f}_1(\tau,z(\tau))+\widetilde{f}_2(\tau,z(\tau))\upsilon(\tau), \qquad 0 \leqslant \tau \leqslant 1, \qquad z(0) = x_0,
    \end{equation}
    где $ \widetilde{f}_1(\tau,z) = \varepsilon f_1(\varepsilon \tau + t_0,z) $, $ \widetilde{f}_2 (\tau,z) = f_2(\varepsilon \tau + t_0,z)$, а управление $ \upsilon(\cdot) $ удовлетворяет ограничениям
    \begin{gather}\label{epscond}
    	\int_0^1 \upsilon^{\top}(\tau) \upsilon(\tau) \, d\tau \leqslant \left( \mu \sqrt{\varepsilon}\right)^2.
    \end{gather}
    Обозначим через $\widetilde{G}(\varepsilon)$ множество достижимости данной системы в момент времени $\tau=1$
    \begin{gather*}
    	\widetilde{G}(\varepsilon)=\{z\in \mathbb{R}^n:\exists v(\cdot)\in \mathbb{L}_2[0,1],  \lVert v(\cdot)\rVert^2_{\mathbb{L}_2[0,1]}
    	\leqslant \left( \mu \sqrt{\varepsilon}\right)^2, \; z=z(1,x^0,v(\cdot))\}.
    \end{gather*}
    По аналогии с \eqref{linear}, линеаризуем \eqref{epsnonlinear} вдоль траектории $ z(\tau,\upsilon(\tau)) = x(\varepsilon \tau + t_0,\varepsilon u(\varepsilon \tau + t_0)) $
    \begin{equation}\label{epslinear}
    	\delta\dot{z} = \varepsilon A(\varepsilon \tau + t_0)\delta z(\tau) + B(\varepsilon \tau + t_0)\delta \upsilon(t),\qquad 0 \leqslant \tau \leqslant 1, \qquad \delta z(0) = 0,
    \end{equation}

	Фундаментальную матрицу $ X_{\varepsilon}(\tau,\xi) $ системы \eqref{epslinear} определим, как решение уравнения
\begin{gather*}
	\frac{dX_{\varepsilon}(\tau,\xi)}{d\tau} = \varepsilon A(\varepsilon \tau + t_0) X_{\varepsilon}(\tau,\xi), \qquad X_{\varepsilon}(\tau,\tau) = I
\end{gather*}

Фундаментальные матрицы систем \eqref{linear} и \eqref{epslinear} эквивалентны с учётом замены времени и обозначения $ \widetilde{X}_{\varepsilon}(\tau,\xi) = X(\varepsilon \tau + t_0,\varepsilon \xi + t_0) $.
\begin{gather*}
	\frac{dX(t,\zeta)}{dt} = A(t) X(t,\zeta), \\
	\zeta = \varepsilon \xi + t_0, \qquad t = \varepsilon \tau + t_0, \\
	\frac{dX(\varepsilon \tau + t_0,\varepsilon \xi + t_0)}{d(\varepsilon \tau + t_0)} = A(\varepsilon \tau + t_0) X(\varepsilon \tau + t_0,\varepsilon \xi + t_0), \\
	\frac{dX_{ \varepsilon}(\tau,\xi)}{d\tau} = \varepsilon A(\varepsilon \tau + t_0) X_{ \varepsilon}(\tau,\xi), \qquad X_{\varepsilon}(\tau,\tau) = I
\end{gather*}

Обозначим через $ \widetilde{W}(\varepsilon) $ грамиан управляемости системы \eqref{epslinear} на отрезке $ [0,1] $. Справедливо следующее утверждение.

\begin{utv}\label{utv}
	{\it Для множеств достижимости систем \eqref{nonlinearT} и \eqref{epsnonlinear} и грамианов управляемости систем \eqref{linear} и \eqref{epslinear} имеют место равенства}
	\begin{gather*}
		\widetilde{G}(\varepsilon)=G(\varepsilon), \qquad \widetilde{G}_y(\varepsilon)=G_y(\varepsilon), \qquad
		\widetilde{W}(\varepsilon) = \dfrac{1}{\varepsilon} W(\varepsilon)
	\end{gather*}
\end{utv}

\doc. 
Действительно, равенства областей достижимости следует из равенств $ x(t_0 + \varepsilon) = z(1) $ и, соответственно, $ y(t_0+\varepsilon) = C z(1) $, где $ x(t) = x(t,u(\cdot)) $, $ y(t) = C x(t,u(\cdot)) $, $ z(\tau) = z(\tau,\nu(\cdot))  $, $ \nu(\tau) = \varepsilon u(\varepsilon \tau + t_0)  $.

Для грамианов управляемости мы имеем
\begin{gather*}
	\begin{gathered}
		\widetilde{W}(\varepsilon) =
		\int_0^1
		X_{ \varepsilon} (1,\tau)
		B(\varepsilon \tau + t_0)
		B^{\top}(\varepsilon \tau + t_0)
		X_{ \varepsilon}^{\top} (1,\tau) \, d\tau = \\
		= \dfrac{1}{\varepsilon}\int_0^1
		X(t_0+\varepsilon,\varepsilon \tau + t_0)
		B(\varepsilon \tau + t_0)
		B^{\top}(\varepsilon \tau + t_0)
		X^{\top}(t_0+\varepsilon,\varepsilon \tau + t_0) \,
		d\left( \varepsilon\tau + t_0\right) = \\ =
		\dfrac{1}{\varepsilon} \int_{t_0}^{t_0+\varepsilon}
		X(t_0+\varepsilon,t)
		B(t)
		B^{\top}(t)
		X^{\top}(t_0+\varepsilon,t) \, dt = \dfrac{1}{\varepsilon} W(\varepsilon). \\ \hfill \square
	\end{gathered}
\end{gather*}


Таким образом грамиан линеаризованной системы с замененным временем \eqref{epslinear} может быть выражен через грамиан линеаризованной в исходном времени системы \eqref{linear}.
    
    
 Как и в предыдущем разделе, определим отображение $S: \mathbb{L}_2[0,1] \rightarrow \mathbb{R}^n $ равенством $S\upsilon(\cdot) = z(1,\upsilon(\cdot))$: здесь $ z(1,\upsilon(\cdot))$ --- решение системы \eqref{nonlinearT}, порожденное управлением $\upsilon(\cdot)$. 
 
 Производная Фреше этого отображения определяется равенством $ S'(\upsilon(\cdot))\delta \upsilon(\cdot) = \delta z(1)$, где $\delta z(t)$ --- решение линеаризованной вдоль пары $\left( z(t,\upsilon(\cdot)),\upsilon(\cdot)\right)  $ системы \eqref{epslinear}, порожденное управлением $\delta \upsilon(\cdot)$ при нулевых начальных условиях.
 
 
 При выполнении условий предположения \ref{Pred} отображение $S'(\upsilon)$ удовлетворяет условию Липшица с константой $L(\varepsilon) = L_0 + L_1\varepsilon$, $ L_0, L_1 \geqslant 0 $. Причем, если коэффициенты матрицы $f_2$ в уравнении системы не зависят от состояния ($f_2(t,x) = f_2(t)$), то $L_0 = 0$. 
  
  Для самосопряженного оператора $S'(0)S'(0)^*$ верно равенство $S'(0)S'(0)^* = \widetilde{W}(\varepsilon)$.
  
   Тогда, управляемость системы \eqref{epslinear} (минимальное собственное число $ \nu(\varepsilon) $ грамиана управляемости $\widetilde{W}(\varepsilon)$ должно быть строго положительно) означает регулярность отображения $S$ при $\upsilon = 0$, а множество достижимости $\widetilde{G}(\varepsilon)$ есть образ гильбертова шара $B_{\mathbb{L}_2}(0,\mu\sqrt{\varepsilon})$ при его отображении $S$.
   
   В данном случае, $S'(0)B_{\mathbb{L}_2}(0,\mu\sqrt{\varepsilon}) = \widetilde{W}^{1/2}(\varepsilon)B_{\mathbb{R}^n}(0,\mu\sqrt{\varepsilon}) $ --- множество достижимости в момент $t = 1$ системы \eqref{epslinear}, линеаризованной вдоль траектории, отвечающей нулевому управлению. Здесь $W^{1/2}(\varepsilon)$ --- арифметический квадратных корень из матрицы $W(\varepsilon)$, $ B_{\mathbb{R}^n}(0,\mu) $ --- евклидов шар радиуса $ \mu $ в $ \mathbb{R}^n $.
 
 Используя теорему \ref{PolyakTh}, сформулируем следующую
 \begin{theorem}\label{ConvexityCond}
 	Множество достижимости $G(\varepsilon)$ системы \eqref{nonlinearT} выпукло, если найдутся такие $K > 0$, $ \alpha > 0$, $ 0 < \varepsilon_0 < \varepsilon$, что для всех $\varepsilon \leqslant \bar{\varepsilon}$
 		\begin{gather}\label{cond}
 			\nu(\varepsilon) \geqslant \left\{ {\begin{array}{*{20}{l}}
 					{K\varepsilon ^{3 - \alpha}, \mbox{\ если \ } f_2(t,x) \mbox{\ не зависит от \ } x}, \\
 					{K\varepsilon ^{1 - \alpha}}, \mbox{\ в противном случае}.
 			\end{array}} \right.
 		\end{gather}
 \end{theorem}
 
    %\subsubsection{Алгоритм построения множеств достижимости методом Монте-Карло}
	%\todo[inline]{Стоит ли это описывать?}
\end{document}